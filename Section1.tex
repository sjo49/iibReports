\documentclass[main.tex]{subfiles}



\begin{document}

\section{Proposal Overview}

The scope of this proposal aims to bridge the gap in communication in the management of knee pain and to enhance patient access to monitored physiotherapy through a wearable device and mobile application.


The market for such wearable devices is lucrative and growing rapidly.
The orthopedic braces and supports market is expected to grow from \$4.1 billion to \$5.3 billion by 2028 globally and to grow by more than \$58 million in the UK alone. (cite these both)
The reason for this rapid growth lies in the ever-increasing number of knee surgeries, particularly in the elderly population, as well as a growing awareness of the benefits of knee braces for injury prevention and rehabilitation.
An additional factor is the growing uptake of sports and fitness activities, which increases the risk of knee injuries and can be a factor in the earlier onset of knee problems.
In 2022, nearly 100,000 primary total knee replacement surgeries were performed in the UK, and this number has been predicted to increase by 85\% by 2030, with one model even predicting over a million annual cases.
Anterior cruciate ligament (ACL) and meniscal repair surgeries have also increased dramatically in the past two decades, with over 30,000 ACL reconstructions performed annually in the UK.
Physiotherapy is vital for the management of knee pain, for post-operation recovery and for re-injury prevention, but is rarely well-monitored and often not followed correctly by patients.

% global knee brace market expected to reach \$1.9 billion by 2025 \cite{infogence}.
% This growth is driven by an increasing number of knee surgeries, particularly in the elderly population, as well as a growing awareness of the benefits of knee braces for injury prevention and rehabilitation.
% Knee braces are commonly used to provide support and stability to the knee joint, particularly after surgery or injury.

The current system for the management of knee injuries and post-surgery rehabilitation is lacking in several key areas, in particular due to insufficient education, guidance and patient accountability. (cite NICE guidelines)
These result in a shortfall of patients receiving first-line treatment (education and physical exercise) for common diagnoses such as osteoarthritis.
In fact, less than 40\% of patients with osteoarthritis recieve such treatment. (https://www.ncbi.nlm.nih.gov/pmc/articles/PMC7990728/)
An evidence review of the management of osteoarthritis by NICE found that patients felt that insufficient information was provided, that direction on self-management techniques was vague and that a lack of monitoring reduced motivation to keep to the program.
These issues are even more severe for vulnerable adults; The NICE review culminated in a recommendation to assess marginal groups, showing how little treatments for knee injuries cater even a little to vulnerable demographics.
However, developing technology for small groups is not economically viable, and so the current system is unlikely to change without first targeting the masses.
The scope of this proposal has therefore been developed to address these wider shortcomings in a way that is scalable and adaptable to vulnerable groups, such as children and adults with learning disabilities.

This proposal details a comfortable knee brace fitted with accelerometers and gyroscopes to monitor the range of motion (ROM) and velocity of movement of the knee joint during a set of exercises.
The brace will be accompanied by a mobile application that will provide interactive education and real-time feedback to the user on their progress through a game interface.
The app will be tailored by physicians to provide a prescribed series of exercises that the user can perform to aid in their recovery.
The user's progress will be tracked over time, providing feedback to the user and their physiotherapist.


\ifSubfilesClassLoaded{%
	\printbibliography{}
}{}


\end{document}