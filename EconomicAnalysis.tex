\documentclass[main.tex]{subfiles}



\begin{document}

\section{Economic Analysis}

\subsection{Market segmentation}
The knee brace market can be segmented into medical and sport applications, as well as by type.
The medical market is driven by the increasing prevalence of knee injuries and osteoarthritis, with the global knee brace market expected to reach \$1.9 billion by 2025.
In the current market, high-quality knee braces can easily range from about £50-500.
A study undertaken with the NHS used braces with an sensor comprised of accelerometers and gyroscopes at a cost of approximately £150 per patient and reduced the average number of post-TKR surgery physiotherapy visits by a third with positive outcomes.



The remote sensor saved the service £1,450 across the pilot

extra £100 to £200 per patient for the sensor
Total physiotherapy visits reduced from 6 to an average of 4

\subsection{Additional sources of support}
Several grants exist in this area to provide support and fund research.
The charity 'Versus Arthritis' has recently awarded almost £100,000 to a team at Leeds University developing an implantable sensor to monitor knee condition following total knee replacement, and almost £2 billion to a sport, exercise and OA centre.
'Power up to Play' is a charity that recently funded a study in Cambridge on warm ups to prevent knee injuries in atheltes.
Charities such as MenCap and the National Lottery Community Fund also provide grants for projects that improve the lives of people with disabilities.
This could be instrumental in adapting commercially successful technology to cater to vulnerable groups that may not provide as large a return.



\ifSubfilesClassLoaded{%
	\printbibliography{}
}{}