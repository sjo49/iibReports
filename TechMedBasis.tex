\documentclass[main.tex]{subfiles}



\begin{document}

\section{Device Design}

\subsection{Knee Brace}

\subsection{Sensor}

A sensor comprising of gyroscopes and accelerometers, known as an Inertial Measurement Unit (IMU), is chosen to monitor the knee brace's orientation and motion over optical tracking or electromyography.
This is because IMUs are small, lightweight and cost-effective, while having been shown to track joint angles and motion

Signal processing for an IMU involves refining data from the accelerometers and gyroscopes to determine orientation and motion.
Filtering removes noise through algorithms like Kalman filters, calibration corrects sensor biases and alignment errors and sensor fusion integrates data from multiple sensors to smooth errors.
This sort of signal processing optimizes IMU data for more precise orientation and motion tracking.

\subsection{}






\ifSubfilesClassLoaded{%
	\printbibliography{}
}{}